\documentclass[letterpaper, twocolumn, 9pt]{article}
\usepackage{url}
\usepackage{hyperref}
\usepackage{array}

%\usepackage{fullpage}
\usepackage{fancybox}


%\newenvironment{comment}[1]{\begin{list}{}{}\item[]{\it #1's Comment:}
%}{{\rm ({\it End of comment.})} \end{list}}

\makeatletter

\setlength\abovecaptionskip{8\p@}
\setlength\belowcaptionskip{-8\p@}

\newcount\@tempcntc
\def\@citex[#1]#2{\if@filesw\immediate\write\@auxout{\string\citation{#2}}\fi
  \@tempcnta\z@\@tempcntb\m@ne\def\@citea{}\@cite{\@for\@citeb:=#2\do
    {\@ifundefined
       {b@\@citeb}{\@citeo\@tempcntb\m@ne\@citea\def\@citea{,}{\bf ?}\@warning
       {Citation `\@citeb' on page \thepage \space undefined}}%
    {\setbox\z@\hbox{\global\@tempcntc0\csname b@\@citeb\endcsname\relax}%
     \ifnum\@tempcntc=\z@ \@citeo\@tempcntb\m@ne
       \@citea\def\@citea{,}\hbox{\csname b@\@citeb\endcsname}%
     \else
      \advance\@tempcntb\@ne
      \ifnum\@tempcntb=\@tempcntc
      \else\advance\@tempcntb\m@ne\@citeo
      \@tempcnta\@tempcntc\@tempcntb\@tempcntc\fi\fi}}\@citeo}{#1}}
\def\@citeo{\ifnum\@tempcnta>\@tempcntb\else\@citea\def\@citea{,}%
  \ifnum\@tempcnta=\@tempcntb\the\@tempcnta\else
   {\advance\@tempcnta\@ne\ifnum\@tempcnta=\@tempcntb \else \def\@citea{--}\fi
    \advance\@tempcnta\m@ne\the\@tempcnta\@citea\the\@tempcntb}\fi\fi}

\def\@listI{
  \leftmargin\leftmargini 
  \parsep 1.8pt plus 2pt minus 1pt
  \topsep 2.5pt plus 2pt minus 2pt
%  \itemsep 1.8pt plus 2pt minus 1pt
  \itemsep 1pt plus 2pt minus 1pt
}
\let\@listi\@listI
\@listi 
\def\@listii{\leftmargin\leftmarginii
  \labelwidth\leftmarginii\advance\labelwidth-\labelsep
  \topsep 1.8pt plus 1.0pt minus 1pt
  \parsep 1.0pt plus 1pt minus 1pt
  \itemsep \parsep
}
\def\@listiii{\leftmargin\leftmarginiii
  \labelwidth\leftmarginiii\advance\labelwidth-\labelsep
  \topsep 1.0pt plus 1pt minus 1pt 
  \parsep \z@ \partopsep 1pt plus 0pt minus 1pt
  \itemsep \topsep
}

\renewenvironment{thebibliography}[1]
     {\section*{\refname
        \@mkboth{\MakeUppercase\refname}{\MakeUppercase\refname}}%
      \list{\@biblabel{\@arabic\c@enumiv}}%
           {\settowidth\labelwidth{\@biblabel{#1}}%
            \leftmargin\labelwidth
            \advance\leftmargin\labelsep
            \@openbib@code
            \usecounter{enumiv}%
            \let\p@enumiv\@empty
            \renewcommand\theenumiv{\@arabic\c@enumiv}}%
      \sloppy\clubpenalty4000\widowpenalty4000%
      \sfcode`\.\@m}
     {\def\@noitemerr
       {\@latex@warning{Empty `thebibliography' environment}}%
      \endlist}

\renewcommand\section{\@startsection {section}{1}{\z@}%
                                   {-2.25ex \@plus -1ex \@minus -.2ex}%
                                   {1.0ex \@plus.2ex}%
                                   {\normalfont\Large\bfseries}}
\renewcommand\subsection{\@startsection{subsection}{2}{\z@}%
                                     {-2.0ex\@plus -1ex \@minus -.2ex}%
                                     {0.75ex \@plus .2ex}%
                                     {\normalfont\large\bfseries}}
\renewcommand\subsubsection{\@startsection{subsubsection}{3}{\z@}%
                                     {-2.0ex\@plus -1ex \@minus -.2ex}%
                                     {0.5ex \@plus .2ex}%
                                     {\normalfont\normalsize\bfseries}}
\renewcommand\paragraph{\@startsection{paragraph}{4}{\z@}%
                                    {1.10ex \@plus1ex \@minus.2ex}%
                                    {-1.25em}%
                                    {\normalfont\normalsize\bfseries}}
\renewcommand\subparagraph{\@startsection{subparagraph}{5}{\parindent}%
                                       {2.00ex \@plus1ex \@minus .2ex}%
                                       {-1em}%
                                      {\normalfont\normalsize\bfseries}}
\makeatother



\newcommand{\comment}[1]{} 

\setlength\pdfpagewidth{8.5in}
\setlength\pdfpageheight{11in}

\topmargin -0.5in
\advance \topmargin by -\headheight
\advance \topmargin by -\headsep
\textheight 9.75in
\oddsidemargin -0.25in
\evensidemargin \oddsidemargin
\marginparwidth 0.5in
\textwidth 7in
\itemsep -1in
\parsep -1in

%%*** abstracts: (Nov. 27, 2009)  //  full papers: (Dec. 04, 2009) ***

\title{\Large\textbf{\textsf{Call for papers \\
12th Workshop on Language Descriptions, Tools, and Applications \\
(LDTA 2012)\\
{\large\url{http://ldta.info}}}}}

\author{~\vspace{-0.4in}\large{During ETAPS (March 31 \& April 1, 2012), in Tallinn, Estonia}}

\date{~\vspace{-0.4in}}

\begin{document}
\maketitle
\thispagestyle{empty}

\paragraph{\textsf{Scope}}

LDTA is an application and tool-oriented workshop focused on
grammarware - software based on grammars in some form. Grammarware
applications are typically language processing applications and
traditional examples include parsers, program analyzers, optimizers
and translators. A primary focus of LDTA is grammarware that is
generated from high-level grammar-centric specifications and thus
submissions on parser generation, attribute grammar systems,
term/graph rewriting systems, and other grammar-related
meta-programming tools, techniques, and formalisms are encouraged.

LDTA is also a forum in which theory is put to the test, in many cases
on real-world software engineering challenges. Thus, LDTA also
solicits papers on the application of grammarware to areas including,
but not limited to, the following:
\begin{itemize}
\item Program analysis, transformation, generation, and verification
\item Implementation of Domain-Specific Languages
\item Reverse engineering and re-engineering
\item Refactoring and other source-to-source transformations
\item Language definition and language prototyping
\item Debugging, profiling, IDE support, and testing
\end{itemize}

Note that LDTA is a well-established workshop similar to other
conferences on (programming)
language engineering topics such as SLE and GPCE, but is solely focused on grammarware.

\paragraph{\textsf{Paper Submission}}
LDTA solicits papers in the following categories:

\begin{itemize}
  \item research papers - original research results within the scope of LDTA
  with a clear motivation, description, analysis, and evaluation.
\item short research papers - new innovative ideas that have not been
  completely fleshed out. As a workshop, LDTA strongly encourages
  these types of submissions.
  experience report papers - description of the use of a grammarware
\item tool or technique to solve a non-trivial applied problem with an
  emphasis on the advantages and disadvantages of the chosen approach
  to the problem.
\item tool demo papers - discussion of a tool or technique that explains
  the contributions of the tool and what specifically will be
  demonstrated. These papers should describe tools and applications
  that do not fit neatly into the specific problems in the Tool
  Challenge.
\end{itemize}

Each submission must clearly state in which of these categories it
falls and not be published or submitted elsewhere. Papers are to use
the standard LaTeX article style and the authblk style for
affiliations; a sample of which is provided at www.ldta.info. Research
and experience papers are limited to 15 pages, tool demonstration
papers are limited to 10 pages, and short papers are limited to 6
pages. The final version of the accepted papers will, pending
approval, be published in the ACM Digital Library and will also be
made available during the workshop.

Please submit your abstract and paper using EasyChair at
\url{http://www.easychair.org/conferences/?conf=ldta2012}.

The authors of each submission are required to give a presentation at
LDTA 2012 and tool demonstration paper presentations are intended to
include a significant live, interactive demonstration.

As for past versions of the workshop, we expect to extend selected
papers to produce journal versions. The authors of the best papers
would be invited to write a journal version of their paper which will
be separately reviewed and, assuming acceptance, be published in
journal form. The publication will most likely be in a special issue
of the journal Science of Computer Programming (Elsevier Science).

\paragraph{\textsf{Invited Speaker}}
To be announced.

\paragraph{\textsf{Important Dates}} ~

\vspace{5pt}
\hspace{-0.20in}
\begin{tabular}{ll}
Abstract submission & Nov. 28, 2011\\
Full paper submission & Dec. 5, 2011\\
Author notification & Jan. 20, 2012\\
Camera-ready papers & Feb. 05, 2012\\
LDTA Workshop & Mar. 31 - Apr. 1, 2012
\end{tabular}

\paragraph{\textsf{LDTA Tools Challenge}}
The 2011 Workshop pioneered the LDTA Tool Challenge where tool
developers were invited to develop solutions to a range of language
processing tasks over a simple but evolving set of imperative
programming languages. We expect a challenge to form part of LDTA
every two years. The 2012 workshop will feature presentations devoted
to a de-brief of the 2011 challenge, based on the paper currently
being prepared by challenge participants.

\newpage

\vspace{3pt}
\hspace{-0.20in}
{\bfseries\textsf{Program Committee}}\\
{\small
\vspace{-9pt}
\begin{tabbing}
 Suzana Andovad (co-chair) \hspace{1.25cm}\= Anya Helene Bagge\\
 Kyung-Goo Doh \> Jeff Gray\\
 Görel Hedin \> Zoltán Horváth\\
 Zhenjiang Hu \> Ivan Kurtev\\  
 Marjan Mernik \> Nate Nystrom\\
 Sylvain Schmitz \> Anthony Sloane (co-chair)\\
 Laurence Tratt \> Vadim Zaytsev \\
 \emph{Full list to be announced later}\\
\end{tabbing}
}
\end{document}
